\documentclass{article}
\usepackage[english]{babel}

\title{20-02}
\author{Denis Turbatu}

\begin{document}
\section*{HEADER}

Header è un contenitore, se metti 'text-align: center;' essendo che l'immagine è contata anche come testo, si allinea centralmente in maniera orizzontale, per la verticalità usi 'line-height' uguale all'altezza dell'header oppure un pò di padding.

Per allineare le cose nell'header per essere tutto sulla stessa linea usi 'vertical-align: middle;'

-----------------------------------------------------------------------------------------------------------------------------------------------------------
BODY

Body è più corretto se contiene il font-family

Meglio evitare il 'height' durante gli esercizi usa di piu padding per spaziare in maniera omogenea tutte le direzioni

'text-align' evidentemente fa a centrare sia l'input che il button,
in quanto gli elementi 'inline-block' sono contati come testo quindi sono soggetti al 'text-align' 

Nell'annidamento di tag simili prova a commentare le sezioni, nell'esercizio avresti potuto commentare la lista grande inizio/fine e commentare inizio/fine delle liste interne

Avresti potuto usare il padding per la lista grande al posto del div centrato

la stessa cosa per gli oggetti della lista, allineavi i 3 span e aggiungevi del padding per far crescere il box degli oggetti, sempre meglio cosi che aumentare il 'line-height'

per i check e i cross conveniva fare una classe cerchio generale e poi una seconda classe check/cross per cambiare il simbolo e il colore ma effettivamente è sempre un picolo cerchio che deve essere uguale per tutti gli oggetti

INVECE per centrare i simboli avrei potuto fare un quadrato, aggiungere il border-radius corretto per il cerchio e fare 'line-height' alle stesse misure del quadrato

fare le targhette inline-block in quanto inline e basta non è soggetto alle altezze e larghezze 

-----------------------------------------------------------------------------------------------------------------------------------------------------------

FLEXBOX

Padre flexbox --> flex container
figli diretti flexbox --> flex items


flex no posiziona ma distribuisce spazio tra gli elementi

display: flex --> cambiamento di natura  |
justify-content --> main axis            |   
align-items --> second axis              |  --> attributi del flex-container
flex-wrap --> quando gli elementi non    |  -->
              entrano nel contenitore    |
              li manda a capo            |

flex-grow --> applica un coefficiente che in base al numero prende in maniera frazionale lo spazio libero del container  | --> attributo del flex-item
 
https://css-tricks.com/snippets/css/a-guide-to-flexbox/
-----------------------------------------------------------------------------------------------------------------------------------------------------------

\end{document}