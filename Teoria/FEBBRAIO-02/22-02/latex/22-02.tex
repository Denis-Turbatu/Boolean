\documentclass{article}

\title{20-02}
\author{Denis Turbatu}
\date{\today}

\begin{document}
\author
\today

\section*{Flex}
Flex è un attributo che gestisce gli spazi tra i contenitori. 
\\[1\baselineskip]
Flex è possibile attribuirlo a due cose, il un contenitore padre che dovrà avere 'display: flex', chiamato flex container
mentre i figli contenuti dal flex container sono chiamati flex items. Essi possono non avere attributi. 
\\[2\baselineskip]
\subsection*{Gap}
La proprietà 'gap' aggiunge uno spazio negli flex items e puo assumere due proprietà 'row' o 'column', un po come margin.
\\[1\baselineskip]
La proprietà 'gap' è un po speciale in quanto aggiunge spazio sia a destra che sinistra quindi:

\title{Margin}
\begin{verbatim}
width: calc(100% / 3 - 10px)
\end{verbatim}

\title{Gap}
\begin{verbatim}
width: calc(100% / 3 - 20 / 3)
\end{verbatim}
\textit{Es: come viene effettuato il codice per l'assegnazione dello spazio con le diverse formule}
\\[2\baselineskip]

\subsection*{Flex-shrink}

Flex-shrink è l'opposto di 'flex-grow', se 'flex-grow' prendeva lo spazio libero e lo assegnava ai flex items, ora 'flex-shrink' toglie lo spazio dei flex items. Questa proprietà ha un coefficiente che più è alto piu un flex item sarà tassato nella sottrazione dello spazio.
\\[2\baselineskip]

\subsection*{Flex-basis}

Flex-basis è la proprietà che assegna un misura che ogni flex item deve avere in direzione sull'asse principale.
\\[2\baselineskip]

\subsection*{Order}

Order è la proprietà che assegna un numero a un flex item in base alla grandezza del numero il posizionamento di un flex itam cambia:
\begin{verbatim}
    -1 --> flex item assegnato prima di tutti
    0 --> flex item nella sua posizione iniziale, coefficiente settato di base
    1 --> flex item assegnato dopo di tutti
    \end{verbatim}



\end{document}